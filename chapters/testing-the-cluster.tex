\chapter{Testing the cluster}

Running applications in Kubernetes means to create some specifications for them. For testing the cluster with my Tarsier application I~will need to have the Consul running at first.

I~have prepared the replication controller configuration for Consul and also the service configuration for it, so it will have the same IP address regardless of which machine it is running on. The Consul will be used for a service discovery among Tarsiers. There is no need to use persistent storage for Consul, holding instances in memory is sufficient enough.

Before Tarsier pod can be created I~have to register secrets for it. There will be two of them. The first secret is a SSL certificate so our service port can be run on HTTPS. The second secret is the Tarsier configuration. It is a good idea to use secrets for configuration because there can be different secrets with different configurations on the development and the production cluster and also we don’t need to rebuild the whole images just because one text file changed.

With the secrets set up, it remains to solve the persistent storage. For logs I~will use the \lstinline{hostPath} volume which is a directory on the physical machine. There will be uniform policy that perhaps in \lstinline{/www/logs/} all pods will store their logs. This way it is provided that logging from the application is fast (to the local drive) and logs will remain there even when the pod dies. Also each pod should use the Kafkafeeder to feed logs to Kafka and delete them after successful upload.
 
For persistent storage plugin of Tarsier I~have to use a real persistent volume. There are many applications in Seznam.cz which need a large amount of data for their start or/and which produce a large amount of data. For those data we have to choose a persistent storage where pod traveling from one machine to another will not be a problem.

I~started 2 GlusterFS \cite{glusterfs} servers and created endpoints and service configuration for them. Now I~can set a volume claim in the Tarsier replication controller configuration and use it without knowing the exact location of the GlusterFS and other paths.

In the Tarsier RC configuration will be more than one container. The second one will be the Kafkafeeder, which also produces logs, so it stores them to \lstinline{/www/logs/kafkafeeder} as well. Then it loads this directory and starts watching it and uploading logs to Kafka. After setting those volumes I~realized that the current usage, where application creates symlink from its log directory to the \lstinline{kafkafeeder.yaml} file in the configuration directory, is no more useful. The kafkafeeder. YAML file which tells the Kafkafeeder where and which logs to upload is stored as another secret in Kubernetes and mounting this file as a volume with logs from the \lstinline{hostPath} is not possible. I~solved this problem by adding another container to the pod, which is created from a simple image with a single static application. User passes the source and the target file to that application and it simply copies it. The snippet of the replication controller is as follows in listing \ref{lst:tarsier-rc}.

With those configurations prepared, the last one missing is the Tarsier service. Then I~will scale the Tarsier to many instances and start sending commands to its interface.

\newpage
\begin{lstlisting}[language=yaml,caption={Snippet of replication controller configuration},label={lst:tarsier-rc}]
containers:
  - name: tarsier
    volumeMounts:
      - name: tarsier-logs
        mountPath: /www/tarsier/logs

  - name: kafkafeeder
    volumeMounts:
      - name: kf-logs
        mountPath: /www/kafkafeeder-beta/logs/

  - name: cp
    args:
      - "/src/kafkafeeder.yaml"
      - "/dst/kafkafeeder.yaml"
    volumeMounts:
      - name: tarsier-kafkafeeder
        mountPath: /src/
      - name: tarsier-logs
        mountPath: /dst/

volumes:
  - name: tarsier-logs
    hostPath: { path: /www/logs/tarsier/ }
  - name: tarsier-kafkafeeder
    secret: { secretName: tarsier-kafkafeeder }
  - name: kf-logs
    hostPath: { path: /www/logs/ }
\end{lstlisting}
